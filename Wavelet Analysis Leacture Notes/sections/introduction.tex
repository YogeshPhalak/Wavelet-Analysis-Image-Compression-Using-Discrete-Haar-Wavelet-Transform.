% !TEX root = ../mat999.tex
\newpage
\section{Introduction \label{sec:intro}}

Signal-Processing is faster and simpler in a sparse representation where few Fourier coefficients reveal the information. Such representations can be constructed by decomposing signals over elementary waveforms chosen in the orthonormal basis domain explained in general Fourier analysis. But results are shown by such methods sometimes introduces errors due to certain conditions. Such conditions can be taken care of by wavelet transforms. 


The discovery of wavelet orthogonal bases and local time-frequency basis has opened the door to huge ways of new transforms. Adapting sparse representations to signal properties, and deriving efficient processing operators, is, therefore, a necessary survival strategy. An orthogonal basis is a dictionary of the minimum size that can yield a sparse representation if designed to concentrate the signal energy over a set of few vectors. This set gives a geometric signal description.